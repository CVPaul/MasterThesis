% created on 2016-03-39
\chapter{结束语}
\label{chap:The_End}
计算视觉问题的研究经过几十年的发展,取得了巨大的成就。在计算机视觉中,集合数据的研究经历十多年时间已然成为视觉任务中的一个热点,其中集合数据主要用图像集合这样一个概念来描述,它有可能是视频、物体的多视角图片、主题相册等。本文的内容主要是针对这样一种集合数据的建模和对应模型下的判别学习方法。

经过10多年的发展,根据图像集合的表示方式的不同,图像集合分类问题相关方法逐渐形成了以下一些类别:1)流形和子空间的方法;2)仿射包相关的方法;3)统计建模的方法;4)深度学习的方法;5)字典学习/稀疏编码的方法等。其中统计建模的方法以其强大的信息编码能力以及简洁的模型表示逐渐发展成为集合数据研究的主要方法之一,同时也由于统计模型的特殊表现形式而需要引入如黎曼流形这样的数学工具对这样一些模型进行研究。而本文也正是在这样的背景下所进行的集合数据建模以及非线性数据结构的判别学习方法的研究。
\section{本文工作总结}
\label{sec:Conclusion}
本文的工作主要围绕集合数据的表示和判别学习展开。首先,作为基础本文在第二章中探讨了矩阵函数的相关问题,并结合学位论文课题中提炼出的相关实例对矩阵流形优化做进行介绍;然后针对使用对称正定矩阵建模图像集合的方法(从最初的协方差矩阵建模图像集合,到后来的高斯模型表示再到最近的GMM模型建模都可以用对称正定矩阵表示)中缺少偏最小二乘回归这样一个强有力的数据分析工具的问题,本文在第三章中提出了黎曼流形上的多切空间偏最小二乘回归方法,并把它用于集合数据的分类问题中。最后,针对二阶统计量表示数据维度过高,样本稀少带来的样本协方差不满秩等问题以及子空间建模没有利用尺度信息(特征值)的问题,并集合着最新的利用固定秩对称半正定矩阵建模图像集合的方法,提出了使用低秩对称半正定矩阵建模图像集合的方法,并进行了实验验证。接下来依次对前几章的内容进行总结说明。

第二章中,我们围绕矩阵函数和流形优化问题进行介绍。在对矩阵函数,流形等基本概念介绍的基础上,针对矩阵流形上的优化问题进行讨论与探究,并结合着从研究生学位论文课题中提炼出的相关实例对矩阵流形优化进行介绍,一方面最终希望帮助读者理解并复现本文提出的方法和结论,另一方面也为解决类似流形优化问题提供借鉴。

第三章在统计模型建模集合数据的大背景下,以黎曼流形为研究工具结合已有工作\cite{PGA,RCCA}研究了黎曼流形中的偏最小二乘问题。该问题的研究中首先参考了\cite{PGA,RCCA}的工作将欧氏空间中的投影的概念泛化到了黎曼流形,并借此定义了黎曼流形上的偏最小二乘的基本版本。后注意到图像集合问题与DTI(Diffusion Tensor Image)研究问题的不同(主要是前者的数据分布更分散也更稀疏)本章在基本版本的流形上的偏最小二乘回归算法的基础上提出了多切空间偏最小二乘回归的方法,在流形的多个切空间进行偏最小二乘问题的学习,并利用逐步回归的思想将学习的结果整合起来;最后以非奇异协方差矩阵即对称正定矩阵黎曼流形为实例,在图像集合问题上实验证明了该方法的有效性,此外文章提出的逐步回归的方案是一个通用的方案,该方案对于其它类型的SPD矩阵流形的表示(如在UIUC\cite{Database_UIUC}数据集上使用的Region Covariance的表示)也是可用的。

第四章的内容是从图像集合的表示的问题的角度出发进行研究的,考虑到使用协方差建模图像集合的时候协方差表示不满秩,协方差矩阵表示维度过高等问题以及子空间建模没有利用尺度信息(特征值)的问题。提出了用低秩半正定矩阵建模图像集合的方法,并针对\cite{PSD_WACV}中使用固定秩的对称半正定矩阵(Fixed-Rank PSD矩阵)流形建模图像集合中固定秩对称半正定矩阵获得方式简单以及距离$\delta_{FRPSD}(\cdot,\cdot)$\ref{polar_metric}割裂了Grassmann流形和SPD矩阵流形之间的关系的问题提出使用编码了判别信息的低秩对称半正定矩阵建模图像集合的方法,并实验验证了该问题,得到了与目前的state-of-the-art可比甚至是稍好一些的结果。

本文的中的内容由图像集合问题衍生出来,更偏向于基础理论,探索了集合数据的建模表示以及非线性结构表示下的判别学习的问题。在此过程中温习原有知识的同时对新的知识也有了更加深入的理解,尤其是矩阵函数相关的问题以及流形上的优化问题,从中获益匪浅。另一方面,这些探索工作也是实验室原有研究方向的延展,期间的尝试有成功也有失败,成功地方希望能够为后来的读者起到参考的意义,而失败的地方也希望读者能够引以为戒。
\section{反思与讨论}
\label{sec:discuss}
本节是对本文中介绍的工作的反思和讨论,主要与第三和第四章的内容相关。通过思考目前存在的一些不足与困扰,期望能对读者有所帮助,以下是具体内容。

在第三章中为了克服基础版本的黎曼流形上的偏最小二乘回归存在的问题提出了使用逐步回归的方法在多个切空间中进行逐步回归学习,从而获得了多且空间偏最小二乘回归算法。这虽然带来了性能上的提升,但是方法中的切空间的个数以及切空间中投影方向数目的选择是两个与算法性能直接相关的参数,究其原因主要是逐步回归的方案的抗过拟合的能力不足导致以上两个参数过大则出现过拟合而过小又会出现欠拟合的问题。因此后续需要考虑更好的多切空间模型融合(整合)的方法。以往的工作\cite{RegionCov_pedestrain}给出了一个很好的启示:使用Adaboost的框架来融合多模型,这样可以有效的控制训练和测试的误差。

在第四章中我们已经指出,虽然提出了使用带判别信息的低秩半正定矩阵来编码label信息,但是编码的框架使用的是相对比较直接的Graph Embedding的框架以及编码过程中由于函数$x^{\frac{1}{n}},n>1$在0点的导数未定义问题使得编码过程中的度量与KDA中度量不一致也对最后的结果造成一定的损失。这部分要求针对PSD矩阵寻找更合适的度量或者散度(如Bregman Divergence)来代替现在的Power Euclidean距离。

最后,注意到Deep Learning已经在各个领域都取得巨大的进展,而在图像集合分类问题中也有一些有益的尝试(如:\cite{Deeplearning_DRM}和\cite{Deeplearning_MMDML}),这些都是初步的尝试,并没有深入的针对图像集合问题进行研究,所以这一部份对接下来的图像集合问题的研究应该是意义重大的。另一个重要的方向是图像集合与静态图像的匹配/分类的问题,这是一个很有应用前景的方向,对检索,追逃等领域有重大意义。