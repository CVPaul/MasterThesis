%%% Local Variables:
%%% mode: latex
%%% TeX-master: "../main"
%%% End:

\begin{ack}
三年的研究生生涯现在走到了尾声,想想当初研究生录用时的喜悦好像又是不久之前的事。研究生三年的时间里获益良多,值此论文付梓之际,希望向所有帮助,支持过我的老师,同学,朋友以及家人表示由衷的感谢。

衷心感谢导师陈熙霖教授将我带入计算所的大门,并为我们的研究与工作提供了高标准的环境。他的指导为我们指出了前进的方向;同时陈老师既是良师也是益友,不仅给予我们学习和研究上的指导,在日常生活中也给予了我们极大的帮助。是陈老师将我带入了计算机视觉的领域,并在这里接触到世界上计算机视觉前沿的研究与工作,开拓了自己的眼界,也让自己的数学背景得以发挥作用。此外,陈老师对科研的热情以及对生活的态度也在潜移默化中改变着自己,他的言传身教将使我终生受益。

诚挚感谢山世光教授的包容与指导;在人脸组的时间,山老师的言传身教给每一位人脸组的同学以极大的鼓舞,山老师的问题往往能一语中的,让人在交谈中豁然开朗。同时山老师对于计算机视觉这个领域的理解和见地也指导着我们的研究与工作,帮助我们拨去心中的疑惑;山老师对别人的包容与理解也给了我们极大的宽慰和鼓舞。山老师以其自身的博学多识,丰富的阅历以及对问题的独到的见解和眼光吸引了一大批优秀的人才,这些优良的品质也是我们学习的榜样和楷模。

由衷的感谢王瑞平副教授的悉心指导和帮助,不管是在生活还是在学习研究上,王老师都给予了我极大的帮助与指导。正是在王老师的指导下我进入本文的主要研究课题,在与王老师的讨论中他对计算机视觉的热情,对于研究的严谨态度以及对于问题的独到的见解都深深的影响着我,让我快速定位问题解决问题的同时也能从问题中获得启示帮助其它研究的推进。同时,王老师对于大方向的把握,长远的目光以及坚定的信念在折服我们的同时为我们的研究工作指明了方向为我们坚定了前进的信念。在生活中王老师亦师亦友,竭尽所能地帮助学生,鼓励学生并且不失幽默风趣,给人一种平易近人的感觉,所以与王老师的相处十分愉快。在科研上,王老师的科研热情和态度,严谨的行事风格以及对于问题的独到见解等都是我们学习的榜样。生活上,王老师以其独特的个人魅力吸引着身边的人,让人愿意与他一起相处共事。

还有很多需要感谢的老师。感谢黄庆明老师,常虹老师,蒋树强老师,苗军老师,蔡秀娟老师,韩琥老师,卿来云老师的教导与解惑,他们的宝贵意见我将终身受用;他们的丰硕的科研成果也让我钦佩万分并给我的研究工作的开展作了重要的启示。感谢实验室办公室的王小彪老师,感谢胡兰平,蔡光辉老师,正是他们的辛勤工作为实验室提供舒适的工作环境,为我们解决了后顾之忧。感谢研究生部周世佳老师,李丹老师,宋守礼老师,张平老师,冯刚老师,李琳老师的默默付出,为我的入学,开题,中期,答辩,就业提供了极大的帮助。

此外还有很多师兄师姐需要感谢,感谢李岩师兄在我刚到实验室的时候帮助迷茫的我排忧解难,他的悉心指导与帮助我度过迷茫的时期。感谢黄智武师兄在研究工作中的指导和帮助,在他的指导和帮助下我得以相对快速的进入研究工作中,并帮助我回到研究的正轨上来。感谢王琪师兄,阚美娜师姐在Intel的凝视矫正项目中的悉心指导和建议以及在平时生活与工作中的帮助,让我在工作与研究中找到平衡并从中学习了做事的方法明白了做研究与做项目的区别。感谢李绍欣师兄,刘昕师兄,王雯师姐,尹芳师姐,刘梦怡师姐,王汉杰师兄,张杰师兄,梁孔明师兄,林宇舜师兄,方正鹏师兄,刘文献师兄,谢广志师兄在我遇到问题时无私的提供帮助。

感谢与我同届的刘昊淼,姜华杰,李振林,李健超,吕雄,邬书哲,邓雪松,叶明全,尹肖贻,张川,许震,杨世杰,王智一,付晓慧几位同学,与他们一起度过了百味的研究生的时光,同他们的交流让我获益匪浅。也要感谢实验室的师弟师妹卢宇衡,乔师师,吴望龙,徐梓宁,何建锋,张梦茹,王芳给实验室注入活力带来了欢乐,也让我反思自身。

感谢 \ucasthesis 的作者朝鲁的无私分享,\ucasthesis 的存在让我的论文写作轻松自在了许多;感谢Intel的刘伟,汤振宇,孙忆晨,郭林楠在我参与Intel凝视矫正项目期间的支持与帮助,从这个项目中学到了很多。

最后,感谢我的家人与朋友是你们在我背后默默的支持着我;虽然求学期间我们聚少离多,但是这并不影响我们之间的关系,正是你们的支持与关怀才让我走到现在,再多的话语也无法表达对你们的感激之情,感谢你们为我所做的一切,我也将竭尽所能回应你们。
\end{ack}
