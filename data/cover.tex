
%%% Local Variables:
%%% mode: latex
%%% TeX-master: t
%%% End:
%\secretcontent{\heiti 绝密}
\secretcontent{}

\ctitle{面向图像集合分类的黎曼流形判别学习方法研究}
% 根据自己的情况选,不用这样复杂
\makeatletter

\makeatother

%\cdegree{{\heiti 工学硕士}}
%\cdepartment[计算所]{{\heiti 中国科学院计算技术研究所}}
%\cmajor{{\heiti 计算机科学与技术}}
%%\cauthor{朝\hspace{1em}鲁} 
%\cauthor{{\heiti 李显求}}
%\csupervisor{{\heiti 陈熙霖\hspace{1em}研究员}}
%\csupervisorplace{{\heiti 中国科学院计算技术研究所}}
\cdegree{工学硕士}
\cdepartment[计算所]{中国科学院计算技术研究所}
\cmajor{计算机科学与技术}
%\cauthor{朝\hspace{1em}鲁} 
\cauthor{李显求}
\csupervisor{陈熙霖\hspace{1em}研究员}
\csupervisorplace{中国科学院计算技术研究所}
% 如果没有副指导老师或者联合指导老师,把下面两行相应的删除即可。


% 日期自动生成,如果你要自己写就改这个cdate
%\cdate{\CJKdigits{\the\year}年\CJKnumber{\the\month}月}
%\cdate{\textbf{\the\year{\heiti 年}\the\month{\heiti 月}}}
\cdate{\the\year 年\the\month 月}

% 博士后部分
% \cfirstdiscipline{计算机科学与技术}
% \cseconddiscipline{系统结构}
% \postdoctordate{2009年7月——2011年7月}

\etitle{Discriminant Learning on Riemannian Manifold for Image Set Classification}
% 这块比较复杂,需要分情况讨论:
% 1. 学术型硕士
%    \edegree:必须为Master of Arts或Master of Science(注意大小写)
%              “哲学、文学、历史学、法学、教育学、艺术学门类,公共管理学科
%               填写Master of Arts,其它填写Master of Science”
%    \emajor:“获得一级学科授权的学科填写一级学科名称,其它填写二级学科名称”
% 2. 学术型博士
%    \edegree:Doctor of Philosophy(注意大小写)
%    \emajor:“获得一级学科授权的学科填写一级学科名称,其它填写二级学科名称”

\edegree{Master of Science}
\eauthor{Li Xianqiu}
\edepartment{Institute of Computing Technology\\Chinese Academy of Sciences}
\emajor{Computer Science and Technology}
\esupervisor{Chen Xilin}

% 这个日期也会自动生成,你要改么?
% \edate{December, 2005}

% 定义中英文摘要和关键字
\begin{cabstract}
视觉作为人类的主要的感知机能之一,对人类感知世界的重要性不言而喻。计算机视觉的任务就是为计算机赋予接近甚至超过人类视觉的感知能力。图像作为计算机视觉任务的主要输入,与其它数据形式(如文本,语音等)相比蕴含了更多的信息。

另一方面尽管图像本身蕴含了丰富的信息但是如何运用这些信息,以及图像本身的一些问题(如视角变化大、光照变化剧烈、分辨率低等)也给视觉任务带来不小的挑战。与此同时,越来越多现实生活中的数据以集合的形式出现:视频监控数据、用户上传视频、主题相册、物体的多视角数据以及动作描述视频等在近年来都呈现出爆发式的增长;图像集合分类问题也在这样的背景下应运而生,针对集合中的数据呈现出的量大但质未必优的特点,图像集合分类问题的核心任务之一便是利用数据量大的特点以克服质低的问题。经过10多年的发展,根据图像集合表示方式的不同,图像集合分类相关方法逐渐形成了以下的一些类别:1、子空间以及流形建模的方法;2、仿射包建模的方法;3、统计建模的方法;4、深度学习的方法;5、其它(稀疏编码,协同表示等)。

在众多方法中,统计建模的方法以其优越表现逐渐成为研究该问题的主要方法之一,本文将以黎曼流形为工具对统计建模图像集合问题进行研究。本文的主要工作包含:1)研究了矩阵函数与流形上的优化理论与方法,在对流形、矩阵函数等概念介绍的基础上,对矩阵流形上的优化问题进行探讨,并结合学位论文课题中的实例对矩阵流形优化进行介绍,一方面帮助读者理解并复现本文所提出的方法,另一方面也为解决类似优化问题提供借鉴。2)提出了黎曼流形上的偏最小二乘回归方法,通过借助切空间构建子流形的方式将欧氏空间中的偏最小二乘回归(Partial Least Square Regression, PLSR)扩展到黎曼流形;并考虑到黎曼流形与欧氏空间的几何结构差异以及图像集合数据稀疏的问题,进一步设计了借助多切空间构建子流形的方法,采用逐步回归的策略整合多个切空间中的结果;本文以非奇异协方差矩阵即对称正定矩阵(Symmetric Positive Definite, SPD)黎曼流形为实例,在集合数据分类问题上进行了实验,取得了与当前最优方法可比甚至更好的结果。3)提出了低秩对称半正定矩阵(Low-Rank symmetric Positive Semi-Definite, PSD)建模图像集合的方法,解决样本协方差矩阵建模图像集合时由于数据稀疏带来的矩阵奇异(不满秩)、由于噪声带来的矩阵估计不准、以及对称正定矩阵表示时空开销大等问题;并采用图嵌入(Graph Embedding)的方法将判别信息内嵌到的低秩对称半正定矩阵表示中,最后在核判别分析(Kernel Discriminant Analysis, KDA)的框架下研究了该表示下的判别学习问题,并验证了低秩对称半正定矩阵表示的有效性。

%  论文的摘要是对论文研究内容和成果的高度概括。摘要应对论文所研究的问题及其研究目
%  的进行描述,对研究方法和过程进行简单介绍,对研究成果和所得结论进行概括。摘要应
%  具有独立性和自明性,其内容应包含与论文全文同等量的主要信息。使读者即使不阅读全
%  文,通过摘要就能了解论文的总体内容和主要成果。
%
%  论文摘要的书写应力求精确、简明。切忌写成对论文书写内容进行提要的形式,尤其要避
%  免“第 1 章……;第 2 章……;……”这种或类似的陈述方式。
%
%  本文介绍中国科学院大学论文模板 \ucasthesis{} 的使用方法。本模板符合学校的硕士、
%  博士论文格式要求。
%
%  本文的创新点主要有:
%  \begin{itemize}
%    \item 用例子来解释模板的使用方法;
%    \item 用废话来填充无关紧要的部分;
%    \item 一边学习摸索一边编写新代码。
%  \end{itemize}
%
%  关键词是为了文献标引工作、用以表示全文主要内容信息的单词或术语。关键词不超过 5
%  个,每个关键词中间用分号分隔。(模板作者注:关键词分隔符不用考虑,模板会自动处
%  理。英文关键词同理。)
\end{cabstract}

\ckeywords{图像集合;统计建模;黎曼流形;判别学习}

\begin{eabstract} 
Vision functionality serves as one of the main abilities for human to percept the real world, and its importance goes without saying. The mission of CV (Computer Vision) is to endow computers with close to or even stronger ability than human to perceive the real world. 

As the main input for CV tasks, images contain much more information than text, audio and so on, but how to make full use of the information becomes a problem. The variations of images bring great challenges to CV tasks. At the same time, data comes more frequently in the form of image set, such as surveillance video, multi-view image sets and so on. Under these background, image set classification comes into being. Image sets usually contain a large amount of images in poor quality. So one major task in image set classification is to overcome the disadvantage of low quality and leverage the advantage of large quantity. 

With more than ten years of development, a lot of methods have been proposed for this task. According to how to model an image set they can be divided into following categories: 1. Subspace/Manifold based methods, 2. Affine hull based methods, 3. Statistics model based methods, 4. Deep Learning based methods. 5. Others, like Dictionary/Sparse coding based method, Collaborative representation methods, etc.

Among the categories listed above, Statistics model based methods have attracted a lot attention with its excellent performance. This thesis takes Riemannian manifold as basic tool and tries to explore statistics model based methods. The main contributions include: 1) Studied matrix function and manifold optimization theory and methods. By introducing the basic concept of manifold and matrix function, along with the real-world problems extracted from the following research topics, optimization algorithms on the manifold have been studied in this thesis (Chapter 2). On the one hand it will help readers understand and implement methods proposed in this thesis, and on the other hand it can also provide basic instructions to solve other similar problems. 2) Proposed Partial Least Square Regression methods on Riemannian manifold with sub-manifold constructed from one tangent space (usually taking the tangent space of samples’ Karcher mean). Then in order to overcome the structure difference between Euclidean space and Riemannian manifold as well as the drawback of sparse sampling, multi-tangent space Partial Least Square Regression method has been designed. On the Symmetric Positive Definite (SPD) matrices manifold, image set classification experiment were designed to evaluate the proposed method and it is observed that the proposed method is comparable or even outperforms the state-of-the-art methods on the commonly used databases. 3) Proposed Low-Rank PSD matrices based image set model to overcome the rank-deficient and high dimension problems of sample covariance models as well as lack of scale information (eigenvalue) drawback in the subspace models. With Graph Embedding framework we encoded label information into Low-Rank PSD (Low-Rank symmetric Positive Semi-Definite) representations of image sets and then designed the discriminant learning methods with Kernel Discriminant Analysis framework. Experiments on the commonly used databases has shown to support our proposition.

%   An abstract of a dissertation is a summary and extraction of research work
%   and contributions. Included in an abstract should be description of research
%   topic and research objective, brief introduction to methodology and research
%   process, and summarization of conclusion and contributions of the
%   research. An abstract should be characterized by independence and clarity and
%   carry identical information with the dissertation. It should be such that the
%   general idea and major contributions of the dissertation are conveyed without
%   reading the dissertation. 
%
%   An abstract should be concise and to the point. It is a misunderstanding to
%   make an abstract an outline of the dissertation and words ``the first
%   chapter'', ``the second chapter'' and the like should be avoided in the
%   abstract.
%
%   Key words are terms used in a dissertation for indexing, reflecting core
%   information of the dissertation. An abstract may contain a maximum of 5 key
%   words, with semi-colons used in between to separate one another.
\end{eabstract}

\ekeywords{Image set, Statistics model, Riemannian manifold, Discriminant learning}
